\def\newex#1{\noindent{\bf #1}}
\newex{1.1} It's \TeX nician.

\newex{2.1} Alice said, ``I always use an en-dash instead of a hyphen
when specifying page numbers like `480--491' in a bibliography.''

\newex{2.2} An em-dash and a hyphen, ----.

\newex{2.3} (Skipped, not enough vocabulary.)

\newex{2.4} ``\thinspace` and `{``}

\newex{2.5} Overloading is always a bad idea.

\newex{3.1} It's {\tt \string\I} and {\tt \string\exercise}.

\newex{3.2} `math\'ematique' and `centim\`etre'

\newex{3.3} Input {\tt \string\show\char`\\}.

\newex{3.4} (TODO)

\newex{4.1} Ulrich Dieter, {\sl Journal f\"ur die reine und angewandte Mathematik\/}
{\bf 201} (1959), 37--70.

\newex{4.2} {\it Add correction\/} before {\it the word}.

\newex{4.3} (TODO)

\newex{4.4} These are not proper control sequences iirc.

\newex{4.5} Rebind {\tt \string\sl} to {\tt \string\cri10}. I'm using a powerful
text editor, so I'd rather fix all occurrences.

\newex{4.6} {\tt at 5pt} and {\tt scaled 500}.

\newex{5.1} shufful shuf{}ful

\newex{5.2} Three {} {} blank spaces in a row.

\newex{5.3} The first, the whole sentence will be centered. The
second, only `S' whill be centered.

\newex{5.4} The whole sentence will be centerized. The {\tt \string\it}
only takes effect on {\it centered}.

\newex{5.5} (TODO)

\newex{5.6} 12345464

\newex{5.7} (TODO)

\newex{6.1} A counter example here, I have typed it three times, but not good at TeX.

\newex{6.2} See what's after {\tt called---}.

\newex{6.3} The heavy bar in printed output.

\newex{6.4} (TODO)

\newex{6.5} By setting {\tt \string\hbadness} to -1.

\newex{6.6} (TODO)

\newex{6.7} The following 5 tokens will be ignored.

\newex{6.8} (See answer)

\newex{7.1} It should be ``Procter \& Gamble's stock climbed to \$2, a 10\% gain''.

\newex{7.2} (See answer)

\newex{7.3} All except 0, 9, 14, 15.

\newex{7.4} Either {\tt \char`\}} or {\tt >}.

\newex{7.5} (See answer), {\tt \string\hi}

\newex{7.6} (TODO)

\newex{7.7} (TODO)

\newex{7.8} Compare {\tt \uppercase{a\lowercase{bC}}} with {\tt ABC}.

\newex{7.9} \copyright\ \uppercase\expandafter{\romannumeral\year}

\newex{7.10} (TODO)

\newex{8.1} {\tt \char`\}} is commented.

\newex{8.2} (See answer) I was wrong with (e).

\newex{8.3} {\sl S}. Space.

\newex{8.4} (See answer)

\newex{8.5} {\tt H(11), i(11), !(12), space(10), \string\par, \string\par}.

\newex{8.6} (See answer)

\newex{8.7} (See answer)

\newex{11.1} Because it's inside a hbox in a hbox.

\newex{11.2} (See answer)

\newex{11.3} {\tt \string\hbox(8.98608+0)x24.44484}

\newex{11.4} (See answer)

\newex{11.5} (TODO)

\newex{11.6} (TODO)

\newex{12.1} 25, 41, 12.

\newex{12.2} The space at the left is twice as much as at the right. The text flush right.

\newex{12.3} (See answer)

\newex{12.4} Mr.~\& Mrs.~User were married by Rev.~Drofnats, who preached on Matt.~19\thinspace:\thinspace3--9.

\newex{12.5} Donald E.\ Knuth, ``Mathematical typography,'' {\sl Bull.\ Amer.\ Math.\ Soc.\ }{\bf 1} (1979), 337--372.

\newex{12.6} Seperate the comma, like {\tt Appendix A\string\hbox\char`\{\char`\}.}, ``Appendix A\hbox{}. Test the effect''.

\newex{13.1} Precede {\tt \string\hbox} with {\tt \string\indent}.

\newex{13.2} (See answer)

\newex{13.3} {\tt \char`\{math shift character \$\char`\}},
{\tt \char`\{restricted horizontal mode: end-group character\char`\}},
{\tt \char`\{\string\bye\char`\}}.

\newex{13.4} (See answer)

\newex{13.5} (TODO)

\newex{16.1} $\gamma+\nu\in\Gamma$

\newex{16.2} $\le \ge \neq$

\newex{16.3} See the difference between $x + _2F_3$ and $x + {}_2F_3$.
The subscript of the previous follow $+$, while the later not.

\newex{16.4} See ${x^y}^z$ and $x^{y^z}$.

\newex{16.5} Because it's ought to be a superscript or subscript.

\newex{16.6} ${{R_i}^{jk}}_l$

\newex{16.7} $10^{10}$\quad$2^{n+1}$\quad$(n+1)^2$\quad
$\sqrt{1-x^2}$\quad$\overline{w+\overline z}$\quad
$p_1^{e_1}$\quad$a_{b_{c_{d_e}}}$\quad$\root 3 \of {h_n''(\alpha x)}$

\newex{16.8} ``If$ x = y$, then $x$ is equal to $y.$'' The spacing and comma.

\newex{16.9} Deleting an element from an $n$-tuple leaves an $(n-1)$-tuple.

\newex{16.10} (See answer)

\newex{16.11} $z^{*2}$, $h_*'(z)$.

\newex{16.12} $3{\cdot}1416$

\newex{16.13} $e^{-x^2}$\quad$D\sim p^\alpha M+l$\quad$\hat g\in(H^{\pi_1^{-1}})'$

\newex{17.1} $x + y^{2/(k+1)}$

\newex{17.2} $((a+1)/(b+1))x$

\newex{17.3} See $${x = (y^2\over k+1)}\hbox{,}$$ to repair, $$x = {y^2\over k+1}\hbox{.}$$

\newex{17.4} {\def\cents{\hbox{\rm\rlap/c}} $7{1\over 2}\cents$}

\newex{17.5} $p^{e'}_2$ is in style $D'$, $e'$ is in style $S'$, 2 is in style $S'$, $'$ is in style $SS'$.

\newex{17.6} $${1\over2}{n\choose k},$$ $$\displaystyle{n\choose k}\over2.$$

\newex{17.7} $${p\choose 2}x^2y^{p-2}-{1\over1-x}{1\over1-x^2}$$

\newex{17.8} $$\sum_{i=1}^p\sum_{j=1}^q\sum_{k=1}^ra_{ij}b_{jk}c_{ki}$$

\newex{17.9} $$\sum_{\scriptstyle1\le i\le p\atop{\scriptstyle1\le j\le q\atop\scriptstyle1\le k\le r}}a_{ij}b_{jk}c_{ki}$$

\newex{17.10}
$$
\biggl({\partial^2\over\partial x^2}+{\partial^2\over\partial y^2}\biggr)
{\bigl|\varphi(x+iy)\bigr|}^2
= 0
$$

\newex{17.11} (See answer)

\newex{17.12} $\bigl(x+f(x)\bigr)\big/\bigl(x-f(x)\bigr)$

\newex{17.13} $$\pi(n) = \sum_{k=2}^n\left\lfloor{\phi(k)\over k-1}\right\rfloor$$

\newex{17.14}
$$
\pi(n)=
\sum_{m=2}^n\left\lfloor\biggl(\sum_{k=1}^{m-1}\bigl\lfloor(m/k)\big/\lfloor m/k\rceil\bigr\rceil\biggr)^{-1}\right\rfloor
$$

\newex{18.1} $R(n, t) = O(t^{n/2})$, as $t \to 0^+$.

\newex{18.2} $$
p_1(n) = \lim_{m\to\infty}\sum_{\nu=0}^\infty\bigl(1-\cos^{2m}(\nu!^n\pi/n)\bigr)
$$

\newex{18.3} (TODO)

\newex{18.4} $x\equiv0 (\pmod y^n)$, the correct one is $x\equiv0\pmod{y^n}$.

\newex{18.5} $$
{n\choose k}\equiv
{\lfloor n/p\rfloor\choose\lfloor k/p\rfloor}
{n \bmod p\choose k\bmod p}
\pmod p.
$$

\newex{18.6} $\bf\bar x^{\rm T}Mx = {\rm 0} \iff x = 0$.

\newex{18.7} $S \subseteq {\mit\Sigma} \iff S \in {\cal S}$

\newex{18.8} $$
{\it available} + \sum_{i=1}^n\max\bigl({\it full}(i), {\it reserved}(i)\bigr)={\it capacity}.
$$

\newex{18.9} (TODO)

\newex{18.10} Let $H$ be a Hilbert space, $C$ a closed bounded convex subset of~$H$, \ $T$~a
nonexpansive self map of~$C$. Suppose that as $n\to\infty$, \ $a_{n,k}\to0$ for each~$k$,
and $\gamma_n = \sum_{k=0}^\infty(a_{n,k+1}-a_{n,k})^+\to 0$. Then for each~$x$ in~$C$, \ 
$A_nx = \sum_{k=0}^\infty a_{n,k}T^k x$ converges weakly to a~fixed point of~$T$.

\newex{18.11} $$
\int_0^\infty{t-ib\over t^2 + b^2}e^{iat}\,dt
= e^{ab}E_1(ab), \qquad a,b > 0.
$$

\newex{18.12} $$
h = 1.0545 \times 10^{-27}\rm\,erg\,sec.
$$

\newex{18.13} (TODO)

\newex{18.14} (TODO)

\newex{18.15} (TODO)

\newex{18.16} $x_1 + x_1x_2 + \cdots + x_1x_2\ldots x_n$, and
$(x_1,\ldots,x_n)\cdot(y_1,\ldots,y_n)=x_1y_1+\cdots+x_ny_n$.

\newex{18.17} See ``Clearly $a_i<b_i$ for~$i=1, 2, \ldots, n$.''
The proper one should be ``Clearly $a_i<b_i$ for~$i=1$,~2, \dots,~$n$.''

\newex{18.18} Well \dots, hardly ever.

\newex{18.19} (TODO)

\newex{18.20} (TODO)

\newex{18.21} $\bigl\{\,x^3\bigm|h(x)\in\{-1, 0, +1\}\,\bigr\}$

\newex{18.22} $\{\,p\mid p$~and $p+2$ are prime$\,\}$

\newex{18.23} $$
f(x) = \cases{
     1/3 & if $0\le x\le1$;\cr
     2/3 & if $3\le x\le4$;\cr
     0 & elsewhere.
}
$$

\newex{18.24} $$
\left\lgroup\matrix{
  a&b&c\cr
  d&e&f\cr
}\right\rgroup
\left\lgroup\matrix{
  u&x\cr
  v&y\cr
  w&z\cr
}\right\rgroup
$$

\newex{18.25} $$\pmatrix{y_1\cr\vdots\cr y_k}$$

\newex{19.1}
$$\sum_{n=0}^\infty a_nz^n\qquad\hbox{converges if}\qquad|z|<\Bigl(\limsup_{n\to\infty}\root n \of{|a_n|}\Bigr)^{-1}$$
$${f(x+\delta x) - f(x)\over \delta x}\to f'(x)\qquad\hbox{as $\delta x\to 0$.}$$
$$\|u_i\|=1,\qquad u_i\cdot u_j=0\quad\hbox{if $i\neq j$.}$$
{\it $$\hbox{The confluent image of}\quad\left\{\matrix{\hbox{an arc}\hfill\cr\hbox{a circle}\hfill\cr\hbox{a fan}\hfill}\right\}
\quad\hbox{is}\quad\left\{\matrix{\hbox{an arc}\hfill\cr\hbox{an arc or a circle}\hfill\cr\hbox{a fan or an arc}\hfill}\right\}.$$}

\newex{19.2} $$\textstyle y={1\over2}x$$

\newex{19.3} Different style, and a mighty influence from spacing.

\newex{19.4} This can be achieved as\vskip 1em
$\displaystyle 1 - {1\over2} + {1\over3} - {1\over4} + \cdots = \ln 2$.

\newex{19.5} $$\prod_{k\ge0}{1\over(1-q^kz)}=\sum_{n\ge0}z^n\bigg/\!\prod_{1\le k\le n}(1-q^k).\eqno{16'}$$

\newex{19.6} $$a=1 \eqno\hbox{(3--1)}$$

\newex{19.7} See $$a=1, \eqno(***)$$ the middle is regarded as binary operation, fix it with
$$a=1. \eqno(*{*}*)$$

\newex{19.8} (TODO)

\newex{19.9} $$\def\clgn{{\lceil\lg n\rceil}}
\eqalign{
T(n)\le T(2^\clgn)&\le c(3^\clgn-2^\clgn)\cr
&<3c\cdot 3^{\lg n}\cr
&=3c\,n^{\lg 3}.\cr
}
$$

\newex{19.10} $$
\eqalign{
P(x)&=a_0+a_1x+a_2x+\cdots+a_nx^n,\cr
P(-x)&=a_0-a_1x+a_2x^2-\cdots+(-1)^na_nx^n.\cr
}
\eqno(30)
$$

\newex{19.11} See $$
\eqalign{
P(x)=a_0+a_1x+a_2x+\cdots+a_nx^n,\cr
P(-x)=a_0-a_1x+a_2x^2-\cdots+(-1)^na_nx^n.\cr
}
\eqno(30)
$$

\newex{19.12} $$
\leqalignno{
\gcd(u, v) &= \gcd(v, u);&(9)\cr
\gcd(u, v) &= \gcd(-u, v).&(10)\cr
}
$$

\newex{19.13} $$
\eqalignno{\left(\int_{-\infty}^\infty e^{-x^2}\,dx\right)^2
&=\int_{-\infty}^\infty\int_{-\infty}^\infty e^{-(x^2+y^2)}\,dx\,dy\cr
&=\int_0^{2\pi}\int_0^\infty e^{-r^2}r\,dr\,d\theta\cr
&=\int_0^{2\pi}\biggl(-{e^{-r^2}\over2}\bigg|_{r=0}^{r=\infty}\biggr)\,d\theta\cr
&=\pi.&(11)\cr
}
$$

\newex{19.14} (TODO)

\newex{19.15} (TODO)

\newex{19.16} (TODO) $$
\displaylines{
\hfill x\equiv x;\hfill\hbox{(1)}\cr
\hfill\hbox{if}\quad x\equiv y\quad\hbox{then}\quad y\equiv x;\hfill\hbox{(2)}\cr
\hfill\hbox{if}\quad x\equiv y\quad\hbox{and}\quad y\equiv z\quad\hbox{then}\quad x\equiv z.\hfill\hbox{(3)}\cr
}
$$

\newex{19.17} $$\eqalignno{
x_nu_1+\cdots+x_{n+t-1}u_t
&=x_nu_1+(ax_n+c)u_2+\cdots\cr
&\qquad+\bigl(a^{t-1}x_n+c(a^{t-2}+\cdots+1)\bigr)u_t\cr
&=(u_1+au_2+\cdots+a^{t-1}u_t)x_n+h(u_1,\ldots,u_t).&(47)\cr
}$$

\bye
